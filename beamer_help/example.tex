\documentclass[16pt]{beamer}
%\mode<presentation>
\usetheme{Berlin}
\setbeamertemplate{footline}[frame number]
\beamertemplatenavigationsymbolsempty

\usepackage[utf8]{inputenc}
\usepackage[ngerman]{babel}
\usepackage[T1]{fontenc}
\usepackage{graphicx}
\usepackage{wrapfig}
\usepackage{listings}
\usepackage{color}

\title{Tutorium 3}
\author{Stefan Heimersheim
	\\\texttt{stefan.heimersheim@rwth-aachen.de}}
\date{13. November 2015}

\lstset{language=C++}
\lstset{showstringspaces=false,keywordstyle=\color{blue}}

\begin{document}
\begin{frame}
\titlepage
\end{frame}



\section{Statische 1D Felder}
\subsection{Überblick}
\begin{frame}
  \begin{itemize}
    \item Mehrere Variablen eines Typs
    \item Sequenz von Daten
    \item Konstante Größe
    \item Intern konstante Zeigern (beschränkte Zeigerarithmetik)
  \end{itemize}
  \begin{itemize}
    \item \texttt{int i[3];}
    \item \texttt{const int n=3;}
    \item \texttt{int i[n];}
    \item \texttt{int i[3]=\{1,2,3\};}
    \item \texttt{i[0]=1;}
  \end{itemize}
\end{frame}
\begin{frame}
  \begin{itemize}
    \item Mehrere Variablen eines Typs
    \item Sequenz von Daten
    \item Konstante Größe
    \item Intern konstante Zeigern (beschränkte Zeigerarithmetik)
  \end{itemize}
  \begin{itemize}
    \item \texttt{int i[3];}
    \item \texttt{const int n=3;}
    \item \texttt{int i[n];}
    \item \texttt{int i[3]=\{1,2,3\};}
    \item \texttt{i[0]=1;}
  \end{itemize}
\end{frame}

\subsection{Hautpspeichermodell}
\begin{frame}
  %\includegraphics[width=0.8\textwidth]{images/speicher1.pdf}
\end{frame}

\section{Zeigerarithmetik}
\subsection{Überblick}
\begin{frame}
  \begin{itemize}
    \item Addition einer ganzen Zahl k auf einen Zeiger p (Zeiger auf eine Variable des Typs T) liefert einen Zeiger gleichen Typs mit einem um $k\cdot sizeof(T)$ (k mal die Größe des von Variablen vom Typ T belegten Speichers in Byte) höheren Wert
    \item Eventuell nützlich für Navigation in Feldern
    \item Unübersichtlicher Code, Zugriff auf unsinnige Bereiche
  \end{itemize}
\end{frame}

\subsection{Beispiel}
\begin{frame}
  %\includegraphics[width=0.8\textwidth]{images/zeigerarith1.pdf}
\end{frame}

\subsection{Erinnerung: Konstanten}
\begin{frame}
  \textbf{Abstimmung: Was ist erlaubt?}
  \\\texttt{int const * b=\&a}%a=*b ist konst
  \begin{itemize}
    \item \texttt{cout << (*b)++;}%Falsch
    \item \texttt{cout << *(b++);}%Richtig
    \item \texttt{cout << *b++;}%Richtig
  \end{itemize}
  Zeigerarithmetik auf Feldern:
  \begin{itemize}
    \item \texttt{cout << *(i+1);} geht
    \item \texttt{i=i+1;} geht nur bei dynamischen Feldern
  \end{itemize}
  Statische Felder sind als konstante Zeiger realisiert
\end{frame}
\begin{frame}
  \textbf{Abstimmung: Was ist erlaubt?}
  \\\texttt{int const * b=\&a}%a=*b ist konst
  \begin{itemize}
    \item \texttt{cout << (*b)++;}%Falsch
    \item \texttt{cout << *(b++);}%Richtig
    \item \texttt{cout << *b++;}%Richtig
  \end{itemize}
  Zeigerarithmetik auf Feldern:
  \begin{itemize}
    \item \texttt{cout << *(i+1);} geht
    \item \texttt{i=i+1;} geht nur bei dynamischen Feldern
  \end{itemize}
  Statische Felder sind als konstante Zeiger realisiert
\end{frame}

\section{Dynamische 1D Felder}
\subsection{Überblick}
\begin{frame}[fragile]
Felder im Heap
  \begin{itemize}
    \item Größe variabel
    \item eigene Speicherverwaltung
    \item delete[] nicht vergessen
  \end{itemize}
  \begin{lstlisting}
  int i=3;
  float* fv=new float[i];
  fv[0]=10;
  delete[] fv;
  \end{lstlisting}
\end{frame}

\subsection{Hautpspeichermodell}
\begin{frame}
 %\includegraphics[width=0.8\textwidth]{images/speicher2.pdf}
\end{frame}

\section{Statische \protect\textit{n}D Felder}
\subsection{Überblick}
\begin{frame}[fragile]
  \begin{itemize}
    \item Mehrdimensionale Felder
    \item Matritzen, Tensoren, ...
  \end{itemize}
  \begin{lstlisting}
  int a[2][3];
  int b[2][3]={{1,2,3},{4,5,6}};
  int c[2][3]={1,2,3,4,5,6};
  int d[2][3]={{1,2},{3,4},{5,6}}; //Fehler
  \end{lstlisting}
\end{frame}

\subsection{Reihenfolge}
\begin{frame}[fragile]
  \begin{lstlisting}
   int a[2][3]={{1,2,3},{4,5,6}};
	cout << &a[0][0] << endl;
	cout << &a[0][1] << endl;
	cout << &a[0][2] << endl;
	cout << &a[1][0] << endl;
	cout << &a[1][1] << endl;
	cout << &a[1][2] << endl;

	0x7fff41d43010
	0x7fff41d43014
	0x7fff41d43018
	0x7fff41d4301c
	0x7fff41d43020
	0x7fff41d43024
  \end{lstlisting}
\end{frame}

\subsection{Hautpspeichermodell}
\begin{frame}
 %\includegraphics[width=0.8\textwidth]{images/speicher3.pdf}
\end{frame}

\section{Dynamische \protect\textit{n}D Felder}
\subsection{Überblick}
\begin{frame}[fragile]
  Beispielcode:
  \begin{lstlisting}
    int** m = new int* [2];
    m[0] = new int[3];
    m[1] = new int[3];
    ...
    delete[] m[0];
    delete[] m[1];
    delete[] m;
  \end{lstlisting}
  Oft in for-Schleifen genutzt:
  \begin{lstlisting}
  int n=2,m=3;
  int** matrix = new int*[n];
  for(int i=0;i<n;i++) matrix[i]=new int[m];
  ...
  for(int i=0;i<n;i++) delete[] matrix[i];
  delete[] matrix;
  \end{lstlisting}
\end{frame}

\subsection{Hauptspeichermodell}
\begin{frame}
 %\includegraphics[width=0.8\textwidth]{images/speicher4.pdf}
\end{frame}


\end{document}
